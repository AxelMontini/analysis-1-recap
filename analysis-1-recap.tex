\documentclass[8pt,a4paper,twocolumn,table]{extarticle}
\usepackage[table]{xcolor}
\usepackage{mathtools}
\usepackage{amsfonts}
\usepackage{amsthm}
\usepackage{amssymb}
\usepackage[margin=0.5cm]{geometry}
\usepackage{hyperref}
\usepackage{todonotes}
\usepackage{xstring}
\usepackage{catchfile}

\setlength{\columnsep}{1cm}
\setlength{\columnseprule}{0.2pt}

\newcommand{\N}{\mathbb{N}}
\newcommand{\R}{\mathbb{R}}
\newcommand{\C}{\mathbb{C}}

\DeclareMathOperator{\acosh}{acosh}
\DeclareMathOperator{\asinh}{asinh}
\DeclareMathOperator{\atanh}{atanh}


\newtheorem{innercustomgeneric}{\customgenericname}
\providecommand{\customgenericname}{}
\newcommand{\newcustomtheorem}[2]{%
  \newenvironment{#1}[1]
  {%
   \renewcommand\customgenericname{#2}%
   \renewcommand\theinnercustomgeneric{##1}%
   \innercustomgeneric
  }
  {\endinnercustomgeneric}
}

\newcustomtheorem{definition}{Definition}

\newcustomtheorem{lemma}{Lemma}

\newcustomtheorem{satz}{Satz}

\newcustomtheorem{korollar}{Korollar}

\newcommand{\seq}[1]{\left( #1_n \right)_{n \ge 1}}

% Read branchname and commit id from HEAD, to be added to the document
\CatchFileDef{\headfull}{.git/HEAD}{}
\StrGobbleRight{\headfull}{1}[\head]
\StrBehind[2]{\head}{/}[\branch]
\IfFileExists{.git/refs/heads/\branch}{%
    \CatchFileDef{\commit}{.git/refs/heads/\branch}{}
}{%
    \newcommand{\commit}{\dots~(in \emph{packed-refs})}
}

\title{Analysis 1 Recap}
\author{Axel Montini \\ \href{mailto:amontini@student.ethz.ch}{amontini@student.ethz.ch}}
\date{\today \hfill \texttt{\branch}, \texttt{\commit}}

\begin{document}
\maketitle

\section{Start}

\begin{korollar}{1.16}
    \label{korollar:inf-sup}
    $A \subseteq B \subseteq \R$, dann
    \begin{itemize}
        \item $B$ nach oben beschraenkt $\implies \sup A \le \sup B$
        \item $B$ nach unten beschraenkt $\implies \inf B \le \inf A$
    \end{itemize}
\end{korollar}

\section{Folgen und Reihen}
\subsection{Folgen}
\begin{definition}{2.4}
    \[
        \left\{ n \in \N : a_n \notin \left] l - \epsilon, l + \epsilon \right[ \right\} \implies \seq{a}\ \mbox{heisst \textbf{konvergent}}
    \]
\end{definition}

\begin{definition}{2.4+}
    Grenzwert oder Limes der Folge: $\underset{n \to \infty}{\lim} a_n$
\end{definition}

\begin{lemma}
    Equivalent Aussagen:
    \begin{enumerate}
        \item $\seq{a}$ konvergiert gegen $l = \lim_{n \to \infty} a_n$
        \item $\forall \epsilon > 0\ \exists N \ge 1\ \left( \lvert a_n - l \rvert < \epsilon \quad \forall n \ge N \right)$
    \end{enumerate}
\end{lemma}

\begin{satz}{2.8}
    Konvergente Folgen $\seq{a},\ \seq{b},\ a = \lim_{n \to \infty} a_n,\ b = \lim_{n \to \infty} b_n$
    \begin{enumerate}
        \item $(a_n + b_n)_{n \ge 1}$ auch konvergent und $a + b = \lim_{n \to \infty} (a_n + b_n)$
        \item $(a_n \cdot b_n)_{n \ge 1}$ konvergiert und $a \cdot b = \lim \dots$
        \item $\forall n \ge 1\ (b_n \ne 0)$ and $b \ne 0$. Dann $\left(\frac{a_n}{b_n}\right)_{n \ge 1}$ konvergent und $\frac{a}{b} = \lim \dots$
        \item $\exists K \ge 1\ \left(\forall n \ge K\ (a_n \le b_n)\right) \implies a \le b$
    \end{enumerate}
\end{satz}

\begin{definition}{2.10}
    \[\forall n \ge 1\ (a_n \le a_{n + 1}) \implies \seq{a} \mbox{ist \textbf{monoton wachsend}}\]
    \[\forall n \ge 1\ (a_n \ge a_{n + 1}) \implies \seq{a} \mbox{ist \textbf{monoton fallend}}\]
\end{definition}

\begin{satz}{2.11}[Weierstrass]
    \begin{itemize}
        \item $\seq{a}$ ist monoton wachsend und nach oben beschränkt. Dann konvergiert es mit \[ \lim_{n \to \infty} a_n = \sup \left\{ a_n : n \ge 1 \right\} \]
        \item $\seq{a}$ ist monoton fallend und nach unten beschränkt. Dann konvergiert es mit \[ \lim_{n \to \infty} a_n = \inf \left\{ a_n : n \ge 1 \right\} \]
    \end{itemize}
\end{satz}

\begin{lemma}{}[Bernoulli Ungleichung]
    $(1 + x)^n \ge 1 + nx\ \forall n \in \N,\ x > -1$
\end{lemma}

\begin{definition}[Limes superior und limes inferior]
    Folge $\seq{a}$ + Weierstrass, definiert man zwei monotone Folgen $\seq{b}$ und $\seq{c}$.
    Fuer jedes $n \ge 1$: $b_n = \inf \left\{ a_k : k \ge n \right\}$ und $c_n = \sup \left\{ a_k : k \ge n \right\}$.
    Dann folgt aus Korollar \ref{korollar:inf-sup}:
    $b_n \le b_{n + 1}\ \forall n \ge 1$ und $c_n \ge c_{n + 1}\ \forall n \ge 1$ (beide konvergieren).

    \begin{itemize}
        \item Limes inferior: $\underset{n \to \infty}{\lim \inf} a_n := \lim_{n \to \infty} b_n$
        \item Limes superior: $\underset{n \to \infty}{\lim \sup} a_n := \lim_{n \to \infty} c_n$
    \end{itemize}\

    Aus $b_n \le c_n$ folgt $\liminf_{n \to \infty} a_n \le \limsup_{n \to \infty} a_n$
\end{definition}

\begin{lemma}{2.19}
    \[
        \seq{a}\ \mbox{konv.} \iff \seq{a}\ \mbox{besch. und}\ \liminf_{n \to \infty} a_n = \limsup_{n \to \infty} a_n
    \]
\end{lemma}

\begin{satz}{2.20}[Cauchy Criterium]
    \label{satz:cauchy-criterium-folge}
    \[ \seq{a}\ \mbox{konvergent} \iff \forall \epsilon > 0\
        \exists N \ge 1\ \left( \forall n,m \ge N \left( \lvert a_n - a_m \rvert < \epsilon \right) \right) \]
\end{satz}

\begin{definition}{2.21}
    \par Abgeschlossenes Intervall (Teilmenge $I \subseteq \R$)
    Form $[a,b]$ oder $[a, +\infty[$ oder $]-\infty, a]$ oder $]-\infty, +\infty[$ mit $a \le b,\ a,b \in \R$.

    Laenge der intervalls: $\mathcal{L}(I) = b - a$ in Fall 1, $\mathcal{L}(I) = +\infty$ anderfalls.
\end{definition}

\begin{satz}{2.25}[Cauchy-Cantor]
    \label{satz:cauchy-cantor}
    $I_1 \supseteq I_2 \supseteq \dots \supseteq I_n \supseteq I_{n + 1} \dots$ Folge abgeschlossener Intervalle mit $\mathcal{L}(I_1) < +\infty$.
    Dann gilt $\bigcap_{n \ge 1} I_n \ne \emptyset$.

    Falls $\underset{n \to \infty}{\lim} \mathcal{L}(I_n) = 0$ enthaelt $\bigcap_{n \ge 1} I_n$ genau einer punkt.
\end{satz}

\begin{definition}{2.27}[Teilfolge]
    Eine Teilfolge von $\seq{a}$ ist $\seq{b}$ wobei $b_n = a_{l(n)}$ und $l: \N^* \to \N^*$ satisfies $l(n) \le l(n + 1)\ \forall n \ge 1$
\end{definition}

\begin{satz}{2.29}[Bolzano-Weierstrass]
    \label{satz:bolzano-weierstrass}
    Jede beschraenkte Folge besitzt eine konvergente Teilfolge.
\end{satz}

\subsection{Reihen}
Sei $\seq{a}$ eine Folge in $\R$ oder $\mathbb{C}$. Der Begriff der Konvergenz der Reihe $\sum_{k = 1}^{\infty} a_k$ stuetzt sich auf die
Folge $\seq{S}$ der Partialsummen $\sum_{k = 1}^n a_k$

\begin{definition}{2.37}
    Die reihe $\sum_{k = 1}^{\infty} a_k$ ist \textbf{konvergent}, falls die Folge $\seq{S}$ der Partialsummen konvergiert.
    Definieren wir $\sum_{k = 1}^{\infty} a_k := \lim_{n \to \infty} S_n$
\end{definition}

\begin{satz}{2.40}
    Seien $\sum_{k=1}^{\infty} a_k$ und $\sum_{j=1}^{\infty} b_j$ konvergent, sowie $\alpha \in \mathbb{C}$.
    Dann:
    \begin{enumerate}
        \item $\sum_{k=1}^{\infty} (a_k + b_k) = \left( \sum_{k=1}^{\infty} a_k \right) + \left( \sum_{j=1}^{\infty} b_j \right)$ ist konvergent.
        \item $\sum_{k=1}^{\infty} \alpha \cdot a_k = \alpha \cdot \sum_{k=1}^{\infty} a_k$ ist konvergent.
    \end{enumerate}
\end{satz}

\begin{satz}{2.41}
    \label{satz:cauchy-criterium-reihe}
    \par Cauchy Criterium, konvergente Reihe
    \[
        \sum_{k=1}^\infty a_k\ \mbox{konvergent}
        \iff
        \forall \epsilon > 0\ \exists N \ge 1\ \left( \left| \sum_{k=n}^m a_k \right| < \epsilon,\ \forall m \ge n \ge N \right)
    \]
\end{satz}

\begin{satz}{2.42}
    \[
        (a_k \ge 0\ \forall k \in \N^*)
        \quad
        \sum_{k=1}^\infty a_k\ \mbox{konvergent}
        \iff
        \seq{S}\ \mbox{nach oben beschr.}
    \]
\end{satz}

\begin{korollar}{2.43}[Vergleichssatz]
    \begin{align*}
        b_k \ge a_k \ge 0\ \forall k \ge 1 \\
        \implies \begin{cases}
            \sum_{k=1}^\infty b_k\ \mbox{konvergent} & \implies \sum_{k=1}^\infty a_k\ \mbox{konvergent} \\
            \sum_{k=1}^\infty a_k\ \mbox{divergent}  & \implies \sum_{k=1}^\infty b_k\ \mbox{divergent}
        \end{cases}
    \end{align*}
\end{korollar}

\begin{definition}{2.45}
    \[ \sum_{k=1}^\infty \left| a_k \right|\ \mbox{konvergiert} \implies  \sum_{k=1}^\infty a_k\ \mbox{\textbf{konvergiert absolut}} \]
\end{definition}

\begin{satz}{2.46}
    Eine absolut kovergente Reihe ist auch konvergent und es gilt $\left| \sum_{k=1}^\infty a_k \right| \le \sum_{k=1}^\infty \left| a_k \right|$
\end{satz}

\begin{satz}{2.48}[Leibniz 1682]
    \begin{align*}
         & \seq{a}\ \mbox{monoton fallend}, a_n \ge 0, \lim_{n \to \infty} a_n = 0                                  \\
         & \implies S := \sum_{k=1}^\infty (-1)^{k+1} a_k \ \mbox{konvergiert und es gilt}\ a_1 - a_2 \le S \le a_1
    \end{align*}
\end{satz}

\begin{definition}{2.50}[Umordnung]
    \begin{align*}
         & \exists \phi: \N^* \to \N^*\ \left( \phi\ \mbox{bijek.,}\ a'_n = a_{\phi(n)} \right) \\
        \implies
         & \sum_{n=1}^\infty a'_n\ \mbox{\textbf{Umordnung} von}\ \sum_{n=1}^\infty a_n
    \end{align*}
\end{definition}

\begin{satz}{2.52}[Dirichlet 1837]
    Falls $\sum_{n=1}^\infty a_n$ absolut konvergiert, denn konvergiert jede Umordnung der Reihe und hat denselben Grenzwert.
\end{satz}

\begin{satz}{2.53}[Quotientenkriterium, Cauchy 1821]
    \begin{align*}
         & \seq{a}, a_n \ne 0\ \forall n \ge 1                                                                           \\
         & \limsup_{n \to \infty} \frac{|a_{n+1}|}{|a_n|} < 1 \implies \sum_{n=1}^\infty a_n\ \mbox{konvergiert absolut} \\
         & \liminf_{n \to \infty} \frac{|a_{n+1}|}{|a_n|} > 1 \implies \sum_{n=1}^\infty a_n\ \mbox{divergiert}
    \end{align*}
\end{satz}

\begin{satz}{2.56}[Wurzelkriterium, Cauchy 1821]
    \begin{align*}
        \limsup_{n \to \infty} \sqrt[n]{|a_n|} < 1 & \implies \sum_{n=1}^\infty a_n\ \mbox{konvergiert absolut}                  \\
        \limsup_{n \to \infty} \sqrt[n]{|a_n|} > 1 & \implies \sum_{n=1}^\infty a_n, \sum_{n=1}^\infty |a_n|\ \mbox{divergieren}
    \end{align*}
\end{satz}

\begin{korollar}{2.57}
    $\sum_{k=0}^\infty c_k z^k$ konvergiert absolut fuer alle $|z| < \rho$ und divergiert fuer alle $|z| > \rho$,
    mit $\rho := \left( \limsup_{k \to \infty} \sqrt[k]{|c_k|} \right)$ (oder $\rho = +\infty$ wenn denominatore 0).
\end{korollar}

\begin{korollar}{}
    Die zeta funktion konvergiert, $\zeta(s) = \sum_{n=1}^\infty \frac{1}{n^s}$
\end{korollar}

\begin{definition}{2.58}
    $\sum_{k=0}^\infty b_k$ ist eine \textbf{lineare Anordnung} der Doppelreihe $\sum_{i,j \ge 0} a_{ij}$ falls es
    eine Bijektion $\sigma : \N \to \N \times \N$ gibt mit $b_k = a_{\sigma(k)}$
\end{definition}

\begin{satz}{2.59}[Cauchy 1821]
    Wir nehmen an dass $\exists B \ge 0$ so dass
    \[ \sum_{i=0}^m \sum_{j=0}^m |a_{ij}| \le B\ \forall m \ge 0 \]
    Dann konvergieren absolut:
    \[
        S_i := \sum_{j=0}^\infty a_{ij}\ \forall i \ge 0,
        U_j := \sum_{i=0}^\infty a_{ij}\ \forall j \ge 0,
        \sum_{i=0}^\infty S_i,
        \sum_{j=0}^\infty U_j,
    \]
    und es gilt $\sum_{j=0}^\infty U_j = \sum_{i=0}^\infty S_i$.
    Jede lineare Anordnung der Doppelreihe konvergiert absolut, mit selbem Grenzwert.
\end{satz}

\begin{definition}{2.60}
    Das \textbf{Cauchy Produkt} der Reihen $\sum_{i=0}^\infty a_i, \sum_{j=0}^\infty b_j$
    ist die Reihe
    \[ \sum_{n=0}^\infty \left( \sum_{j=0}^\infty a_{n-j} \cdot b_j \right) = a_0b_0 + (a_0b_1 + a_1b_0) + (a_0b_2 + a_1b_1 + a_2b_0) + ... \]
\end{definition}
\begin{satz}{2.62}
    Falls zwei Reihen absolut konvergieren, so konvergiert ihr Cauchy Produkt und es gilt
    \[ \sum_{n=0}^\infty \left( \sum_{j=0}^\infty a_{n-j} \cdot b_j \right) = \left( \sum_{i=0}^\infty a_i \right) \left( \sum_{j=0}^\infty b_j \right)\]
\end{satz}

\section{Stetige Funktionen}
\begin{definition}{3.1}
    Sei $f \in \R^D$.
    $f$ ist nach \textbf{(oben/unten) beschränkt}, falls $f(D) \subseteq \R$ nach (oben/unten) beschränkt ist.
\end{definition}

\begin{definition}{3.2}
    Eine Funktion $f : D \to \R,\ D \subseteq \R$ ist
    \begin{enumerate}
        \item \textbf{monoton wachsend}, falls $\forall x,y \in D\ \left( x \le y \implies f(x) \le f(y) \right)$
        \item \textbf{streng m. wachsend}, falls $\forall x,y \in D\ \left( x < y \implies f(x) < f(y) \right)$
        \item \textbf{monoton fallend}, falls $\forall x,y \in D\ \left( x \le y \implies f(x) \ge f(y) \right)$
        \item \textbf{streng m. fallend}, falls $\forall x,y \in D\ \left( x < y \implies f(x) > f(y) \right)$
        \item \textbf{monoton}, falls $f$ monoton wachsend oder monoton fallend ist.
        \item \textbf{streng monoton}, falls $f$ streg monoton wachsend oder streng monoton fallend ist.
    \end{enumerate}
\end{definition}

\begin{definition}{3.4}
    Sei $D \subseteq \R,\ x_o \in D$. Die funktion $f: D \to \R$ ist \textbf{in $x_0$ stetig}, falls
    \[ \forall \epsilon > 0\ \exists \delta > 0\ \forall x \in D\ \left( \left| x - x_0 \right| < \delta \implies \left| f(x) - f(x_0) \right| < \epsilon \right) \]
\end{definition}

\begin{definition}{3.5}
    Die Funktion $f: D \to \R$ ist \textbf{stetig}, falls sie in jedem Punkt von $D$ stetig ist.
\end{definition}

\begin{satz}{3.7}
    Sei $x_0 \in D \subseteq \R$ und $f: D \to \R$.
    $f$ ist \textit{genau dann} in $x_0$ stetig, falls für jede Folge $\seq{a}$ in $D$ folgende Implikation gilt:
    \[ \lim_{n \to \infty} a_n = x_0 \implies \lim_{n \to \infty} f(a_n) = f(x_0) \]
\end{satz}

\begin{definition}{3.9}
    Eine \textbf{polynomiale Funktion} $P: \R \to \R$ ist eine Funktion der Form
    \[ P(x) = a_n x^n + ... + a_0, \quad a_0, ..., a_n \in \R \]
    Falls $a_n \ne 0$ ist $n$ der \textbf{Grad} von $P$
\end{definition}

\begin{korollar}{3.11}
    Seien $P, Q \ne \mathbf{0}$ poly. Funk. auf $\R$. Seien $x_1, ..., x_m$ die Nullstellen von $Q$. Dann ist die folgende Funktion stetig.
    \[ \frac{P}{Q}: \R \setminus \left\{ x_1, ..., x_m \right\} \to \R,\quad x \mapsto \frac{P(x)}{Q(x)} \]
\end{korollar}

\begin{satz}{3.12}[Zwischenwertsatz]
    Sei $I \subseteq \R$ ein Intervall, $f: I \to \R$ eine stetige Funktion und $a, b \in R$.
    \[ \forall c \in \R \ \exists z \in I\ \left( f(a) \le c \le f(b) \implies a \le z \le b \right) \]
\end{satz}

\begin{korollar}{3.13}
    Sei $P(x) = a_nx^n + ... + a_0$ ein Polynom mit $a_n \ne 0$ und $n$ ungerade.
    Dann besitzt $P$ mindestens eine Nullstelle in $\R$.
\end{korollar}

\begin{definition}{3.16}
    Ein intervall $I \subseteq \R$ ist \textbf{kompakt}, falls es von der Form $I = [a, b],\ a \le b$ ist.
\end{definition}

\begin{satz}{3.19}[Min-Max] Sei $f: I = [a,b] \to \R$ stetig auf einem kompakten Intervall $I$.
    Dann gibt es $u,v \in I$ mit
    \[ f(u) \le f(x) \le v \quad \forall x \in I \]
    Insbesondere ist f beschränkt
\end{satz}

\begin{satz}{3.20}[Umkehrabbildung]
    Seien $D_1,D_2 \subseteq \R$ zwei Teilmenger, $f: D_1 \to D_2$, $g: D_2 \to \R$ Funktionen und $x_0 \in D_1$.
    Falls $f$ in $x_0$ und $g$ in $f(x_0)$ stetig sind, so ist $g \circ f: D_1 \to \R$ in $x_0$ stetig
\end{satz}

\begin{satz}{3.22}
    Sei $I \subseteq \R$ ein Intervall und $f: I \to \R$ stetig, streng monoton.
    Dann ist $J := f(I) \subseteq \R$ ein Intervall und $f^{-1}: J \to I$ ist stetig, streng monoton.
\end{satz}

\begin{satz}{3.24}[Exponentialfunktion]
    $\exp: \R \to ]0,+\infty[$ ist streng monoton wachsend, stetig und surjektiv.
\end{satz}

\begin{korollar}{3.27}
    \[
        \exp(x) \ge 1 + x\ \forall x \in \R\ \mbox{und}\
        \exp(x) = \lim_{n \to \infty} \left( 1 + \frac{x}{n} \right)^n \]
\end{korollar}

\begin{korollar}{3.28}
    $\ln: ]0,+\infty[ \to \R$ ist eine streng monoton wachsende, stetige, bijektive Funktion.
            $\ln(a \cdot b) = \ln(a) + \ln(b)\quad \forall a,b \in ]0,+\infty[$
\end{korollar}

\subsection{Konvergenz von Funktionenfolgen}
\begin{definition}{3.30}[Konvergenz von Funktionenfolgen]
    Die Funk.folge $(f_n)_{n \ge 0}$ \textbf{konvergiert punktweise} gegen eine Funktion $f: D \to \R$, falls für alle $x \in D$:
    \[ f(x) = \lim_{n \to \infty} f_n(x) \]
\end{definition}

\begin{definition}{3.32}[Weierstrass 1841]
    Die folge $f_n: D \to \R$ \textbf{konvergier gleichmässig} in $D$ gegen $f: D \to \R$ falls gilt
    \[ \forall \epsilon > 0\ \exists N \ge 1\ (\forall n \ge N\ \forall x \in D : |f_n(x) - f(x)| < \epsilon ) \]
\end{definition}

\begin{satz}{3.33}
    Sei $D \subseteq \R$ und $f_n : D \to \R$ eine Funktionenfolge bestehend aus in $D$ stetigen Funktionen die in $D$ gleichmässig eine FUnktion $f: D \to \R$ konvergiert.
    Dann ist $f$ in $D$ auch stetig.
\end{satz}

\begin{definition}{3.34}
    Eine Funktionenfolge $f_n$ ist \textbf{gleichmässig konvergent} falls für alle $x \in D$ der Grenzwert $f(x) := \lim_{n \to \infty} f_n(x)$
    existiert und die Folge $(f_n)_{n \ge 0}$ gleichmässig gegen $f$ konvergiert.
\end{definition}

\begin{definition}{3.37}
    Die Reihe $\sum_{k=0}^\infty f_k(x)$ konvergiert gleichmässig in $D$ falls die durch $S_n(x) = \sum_{k=0}^n f_k(x)$ definierte
    Funktionenfolge gleichmässig konvergiert.
\end{definition}

\begin{satz}{3.38}
    Sei $D \subseteq \R$ und $f_n : D \to \R$ eine Folge stetiger Funktionen. Wir nehmen an, dass $|f_n(x)| \le c_n\ \forall x \in D$
    und, dass $\sum_{n=0}^\infty c_n$ konvergiert. Dann konvergiert die Reihe $\sum_{n=0}^\infty f_n(x)$ gleichmässig in $D$ und deren Grenzwert
    \[ f(x) := \sum_{n=0}^\infty f_n(x) \]
    ist eine in $D$ stetige Funktion.
\end{satz}

\begin{definition}{3.39}
    Die Potenzreihe $\sum_{k=0}^\infty c_k x^k$ hat \textbf{positiven Konvergenzradius} falls $\underset{k \to \infty}{\limsup} \sqrt[k]{|c_k|}$ existiert.
    Der Konvergenzradius ist dann definiert als:
    \begin{align*}
        \rho = \begin{cases}
            +\infty                                                    & \mbox{falls}\ \underset{k \to \infty}{\limsup} \sqrt[k]{|c_k|} = 0 \\
            \frac{1}{\underset{k \to \infty}{\limsup} \sqrt[k]{|c_k|}} & \mbox{falls}\ \underset{k \to \infty}{\limsup} \sqrt[k]{|c_k|} > 0
        \end{cases}
    \end{align*}
\end{definition}

\begin{satz}{3.40}
    Sei $\sum_{k=0}^\infty c_k x^k$ eine Potenzreihe mit positiven Konvergenzradius $\rho > 0$ und sei
    $f(x) := \sum_{k = 0}^\infty c_k x^k,\ |x| < \rho$.

    Dann gilt: $\forall 0 \le r < \rho$ konvergiert
    \[ \sum_{k=0}^\infty c_k x^k \]
    gleichmässig auf $[-r, r]$, insbesondere ist $f: ]-\rho, \rho[ \to \R$ stetig.
\end{satz}

\subsection{Trigonometrische Funktionen}

\begin{satz}{3.42}
    \begin{enumerate}
        \item $\exp iz = \cos(z) + i \sin(z)\quad \forall z \in \C$
        \item $\cos z = \cos(-z)$ und $\sin(-z) = \sin(z)\quad \forall z \in C$
        \item $\sin z = \frac{e^{iz} - e^{-iz}}{2i},\ \cos z = \frac{e^{iz} + e^{-iz}}{2}$
        \item $\sin(z + w) = \sin(z)\cos(w) + \cos(z)\sin(w)$ und $\cos(z + w) = \cos(z)\cos(w) - \sin(z)\sin(w)$
        \item $\cos(z)^2 + \sin(z)^2 = 1\quad \forall z \in \C$
    \end{enumerate}
\end{satz}

\begin{korollar}{3.43}
    \begin{align*}
         & \sin(2z) = 2\sin(z)\cos(z)       \\
         & \cos(2z) = \cos(z)^2 - \sin(z)^2
    \end{align*}
\end{korollar}

\begin{satz}{3.44}
    $\pi := \left\{ t > 0 : \sin t = 0 \right\}$ \dots
\end{satz}

\begin{satz}{}
    \begin{alignat*}{2}
        \sin(x) & = \sum_{n = 0}^\infty \frac{(-1)^n x^{2n + 1}}{(2n + 1)!} & = x - \frac{x^3}{3!} + \frac{x^5}{5!} - \frac{x^7}{7!} + ... \\
        \cos(x) & = \sum_{n = 0}^\infty \frac{(-1)^n x^{2n}}{(2n)!}         & = 1 - \frac{x^2}{2!} + \frac{x^4}{4!} - \frac{x^6}{6!} + ...
    \end{alignat*}
\end{satz}

\subsection{Grenzwerte}

\begin{definition}{3.47}
    $x_0 \in \R$ ist ein \textbf{Häufungspunkt} der Menge $D$ falls $\forall \delta > 0$:
    \[ (]x_0 - \delta, x_0 + \delta[ \setminus \{x_0\}) \cap D \ne \emptyset \]
\end{definition}

\begin{definition}{3.49}
    Sei $f: D \to \R$, $x_0 \in \R$ ein Häufungspunkt von $D$. Dann ist $A \in \R$ der Grenzwert von $f(x)$ für $x \to x_0$, bezeichnet mit
    \[ " \lim_{x \to x_0} f(x) = A " \]
    falls $\forall \epsilon > 0\ \exists \delta > 0$ so dass
    \[ \forall x \in D \cap \left(  ]x_0 - \delta, x_0 + \delta[ \setminus \{x_0\}  \right) : |f(x) - A| \le \epsilon \]
\end{definition}

\begin{satz}{3.52}
    $D, E \subseteq \R$, $x_0$ Häufungspunkt von $D$, $f: D \to E$ eine Funktion, $y_0 := \lim_{x \to x_0} f(x)$ existiert und $g: E \to \R$ stetig in $y_0$ ist.
    Dann \[ \lim_{x \to x_0} g(f(x)) = g(y_0) \]
\end{satz}

\begin{definition}{}[Linksseitige und rechtsseitige Grenzwerte]
    \[ \lim_{x \to x_0^+} f(x) \quad \forall \epsilon\ \exists \delta,\ \forall x \in D \cap ]x_0, x_0 + \delta[: \dots \]
    Analogous for $\lim_{x \to x_0^-} f(x)$.
\end{definition}

\section{Differenzierbare Funktionen}

\begin{definition}{4.1}
    $f$ ist \textbf{in $x_0$ differenzierbar} falls der Grenzwert
    \[ f'(x_0) = \lim_{x \to x_0} \frac{f(x) - f(x_0)}{x - x_0} \]
    existiert.
\end{definition}

\begin{satz}{4.3}[Weierstrass 1861]
    Sei $f: D \to \R$, $x_0 \in D$ Häufungspunkt von $D$. Folgende Aussagen sind äquivalent:
    \begin{enumerate}
        \item $f$ ist in $x_0$ differenzierbar
        \item Es gibt $c \in \R$ und $r: D \to \R$ mit:
              \begin{enumerate}
                  \item $f(x) = f(x_0) + c(x - x_0) + r(x)(x - x_0)$
                  \item $r(x_0) = 0$ und $r$ ist stetig in $x_0$.
              \end{enumerate}
    \end{enumerate}

    Falls dies zutrifft ist $c = f'(x_0)$ eindetig bestimmt
\end{satz}

\begin{satz}{4.7}
    $f: D \to \R$ ist \textbf{in $D$ differenzierbar}, falls für jeden Häufungspunkt $x_0 \in D$, $f$ in $x_0$ differenzierbar ist.
\end{satz}

\begin{satz}{4.9}
    $D \subseteq \R$, $x_0 \in D$ Häufungspunkt von $D$ und $f,g: D \to \R$ in $x_0$ differenzierbar.
    Dann
    \begin{enumerate}
        \item $f+g$ in $x_0$ differenzierbar und $(f + g)'(x_0) = f'(x_0) + g'(x_0)$
        \item $f \cdot g$ in $x_0$ differenzierbar und $(f \cdot g)'(x_0) = f'(x_0)g(x_0) + f(x_0)g'(x_0)$
        \item Falls $g(x_0) \ne 0$ ist $\frac{f}{g}$ in $x_0$ differenzierbar und
              $\left( \frac{f}{g} \right)'(x_0) = \frac{f'(x_0)g(x_0) - f(x_0)g'(x_0)}{g(x_0)^2}$
    \end{enumerate}
\end{satz}

\begin{satz}{4.11}
    Seien $D, E \subseteq R$ und sei $x_0 \in D$ ein Häufungspunkt. Sei $f: D \to E$ in $x_0$ differenzierbar
    so dass $y_0 := f(x_0)$ ein Häufungspunkt von $E$ ist und sei $g: E \to \R$ in $y_0$ differenzierbar.

    Dann ist $g \circ f: D \to \R$ in $x_0$ differenzierbar und
    \[ (g \circ f)'(x_0) = g'(f(x_0))f'(x_0) \]
\end{satz}

\subsection{Zentrale Sätze}

\begin{definition}{4.14}
    $f: D \to \R$, $D \subseteq \R$ und $x_0 \in D$
    \begin{enumerate}
        \item $f$ besitzt ein lokales Maximum in $x_0$ falls
              \[
                  \exists \delta > 0\ \forall x \in ]x_0 - \delta, x_0 + \delta[ \cap D\ \left( f(x) \le f(x_0) \right)
              \]
        \item $f$ besitzt ein lokales Minimum in $x_0$ falls
              \[
                  \exists \delta > 0\ \forall x \in ]x_0 - \delta, x_0 + \delta[ \cap D\ \left( f(x) \ge f(x_0) \right)
              \]
        \item $f$ besitzt ein \textbf{lokales Extremum} in $x_0$ falls es entweder ein lokales Minimum oder Maximum von $f$ ist.
    \end{enumerate}
\end{definition}

\begin{satz}{4.16}[Rolle 1690]
    $f: [a, b] \to \R$ stetig und in $]a, b[$ differenzierbar.
            \[ f(a) = f(b) \implies \exists \xi \in ]a, b[\ \left( f'(\xi) = 0 \right) \]
\end{satz}

\begin{satz}{4.17}[Lagrange 1797]
    $f: [a, b] \to \R$ stetig und in $]a, b[$ differenzierbar. Dann gibt es $\xi \in ]a, b[$ mit
    \[ f(b) - f(a) = f'(\xi)(b - a) \]
\end{satz}

\begin{satz}{4.22}[Cauchy]
    $f,g : [a,b] \to \R$ stetig und in $]a, b[$ differezierbar.
                    Dann gibt es $\xi \in ]a, b[$ mit
                    \[ g'(\xi)(f(b) - f(a)) = f'(\xi)(g(b) - g(a)) \]

                    falls $g'(x) \ne 0\ \forall x \in ]a, b[$ folgt
    \begin{align*}
         & g(a) \ne g(b)                                             \\
         & \frac{f(b) - f(a)}{g(b) - g(a)} = \frac{f'(\xi)}{g'(\xi)}
    \end{align*}
\end{satz}

\begin{satz}{4.23}[l'Hôpital 1696]
    $f,g: ]a, b[ \to \R$ differenzierbar mit $g'(x) \ne 0\ \forall x \in ]a, b[$.
    Falls
    \[ \lim_{x \to b^-} f(x) = 0,\quad \lim_{x \to b^-} g(x) = 0 \]
    und
    \[ \lim_{x \to b^-} \frac{f'(x)}{g'(x)} =: \lambda \]
    existiert, folgt
    \[ \lim_{x \to b^-} \frac{f(x)}{g(x)} = \lambda \]
\end{satz}

\begin{definition}{4.26}
    \begin{enumerate}
        \item $f$ ist \textbf{konvex} auf $I$ falls für alle $x \le y,\ x,y \in I$ und $\lambda \in [0,1]$
              \[ f(\lambda x + (1 - \lambda)y) \le \lambda f(x) + (1 - \lambda)f(y) \]
              gilt.
        \item $f$ ist \textbf{streng konvex} auf $I$ falls für alle $x < y,\ x,y \in I$ und $\lambda \in [0,1]$
              \[ f(\lambda x + (1 - \lambda)y) < \lambda f(x) + (1 - \lambda)f(y) \]
              gilt.
    \end{enumerate}
\end{definition}

\begin{satz}{4.29}
    $f: ]a, b[ \to \R$ in $]a, b[$ differenzierbar. $f$ ist genau dann (streng) konvex, falls $f'$ (streng) monoton wachsend ist.
\end{satz}

\subsection{Höhere Ableitungen}

\begin{definition}{4.32}
    \begin{enumerate}
        \item Für $n \ge 2$ ist $f$ \textbf{n-mal differenzierbar in $D$} falls $f^{(n-1)}$ in $D$ differenzierbar ist.
              Dann ist $f^(n) := (f^{(n - 1)})'$ und nennt sich die n-te Ableitung von $f$.
        \item Die funktion $f$ ist \textbf{n-mal stetig differenzierbar in $D$}, falls die n-mal differenzierbar ist und falls $f^{(n)}$ in $D$ stetig ist.
        \item Die funktion $f$ ist in $D$ \textbf{glatt}, falls sie $\forall n \ge 1$ n-mal differenzierbar ist.
    \end{enumerate}
\end{definition}

\begin{satz}{4.34}
    $f,g: D \to \R$ n-mal differenzierbar in $D$, dann:
    \begin{enumerate}
        \item $f + g$ n-mal diff. ist und $(f + g)^{(n)} = f^{(n)} + g^{(n)}$
        \item $f \cdot g$ n-mal diff. ist und
              \[ (f \cdot g)^{(n)} = \sum_{k=0}^n \binom{n}{k} f^{(k)}g^{(n - k)} \]
    \end{enumerate}
\end{satz}

\subsection{Taylor}\todo[inline]{Maybe?}

\section{Riemann Integral}

\begin{definition}{5.1}
    Eine \textbf{Partition} von $I$ ist eine endliche Teilmenge $P \subsetneq [a, b]$ wobei $\{a,b\} \subseteq P$.
\end{definition}

\begin{definition}{}[Untersumme, Obersumme]
    \begin{alignat*}{2}
        \mbox{Partition}\  & P = \{x_0, x_1, ..., x_n\}\           & \mbox{mit}\ x_0 = a < x_1 < ... < x_n = b \\
                           & \delta_i = x_i - x_{i - 1},           & i \ge 1                                   \\
                           & s(f, P) := \sum_{i=1}^n f_i \delta_i, & f_i = \inf_{x_{i - 1} \le x \le x_i} f(x) \\
                           & S(f, P) := \sum_{i=1}^n F_i \delta_i, & F_i = \sup_{x_{i - 1} \le x \le x_i} f(x) \\
    \end{alignat*}
\end{definition}

\begin{lemma}{5.2}
    \begin{enumerate}
        \item Sei $P'$ eine Vereinfachung von $P$, dann gilt $s(f, P) \le s(f, P') \le S(f, P') \le S(f, P)$
        \item Für beliebige Partitionen $P_1, P_2$ gilt: $s(f, P_1) \le S(f, P_2)$
    \end{enumerate}
\end{lemma}

\begin{definition}{}
    Sei $\mathcal{P}(I)$ der Menge der Partitionen von $I$.
    \begin{align*}
        s(f) & := \sup_{P \in \mathcal{P}(I)} s(f, P) \\
        S(f) & := \inf_{P \in \mathcal{P}(I)} S(f, P)
    \end{align*}
\end{definition}

\begin{satz}{5.3}
    Eine beschränkte Funktion $f: [a,b] \to \R$ ist \textbf{Riemann integrierbar} (oder kurz: integrierbar) falls
    \[ s(f) = S(f) \]
    In diesem fall bezeichnen wir den gemeinsamen Wert von $s(f)$ und $S(f)$ mit
    \[ \int_a^b f(x) dx \]
\end{satz}

\begin{satz}{5.4}
    Eine beschränkte Funkiton ist genau dann integrierbar falls
    \[ \forall \epsilon > 0\ \exists P \in \mathcal{P}(I)\ \mbox{mit}\ S(f, P) - s(f, P) \le \epsilon \]
\end{satz}

\begin{satz}{5.8}[Du Bois-Reymond 1875, Darboux 1875]
    Eine beschränkte Funktion $f: [a, b] \to \R$ ist genau dann integrierbar, falls $\forall \epsilon > 0\ \exists \delta > 0$
    so dass
    \[ \forall P \in \mathcal{P}_\delta(I),\ S(f, P) - s(f, P) < \epsilon \]
\end{satz}

\begin{satz}{5.10}
    Seien $f, g: [a, b] \to \R$ beschränkt, integrierbar und $\lambda \in R$. Dann sind $f + g$, $\lambda \cdot f$, $f \cdot g$, $|f|$,
    $\max(f, g)$, $\min(f, g)$ und $\frac{f}{g}$ (falls $|g(x)| \ge \beta > 0\ \forall x \in [a, b]$) integrierbar
\end{satz}

\begin{definition}
    Eine funktion $f: D \to \R$ ist \textbf{gleichmässig stetig}, falls
    \[ \forall \epsilon > 0\ \exists \delta > 0\ \forall x,y \in D\ \left( |x - y| < \delta \implies |f(x) - f(y)| < \epsilon \right) \]

    Example: $f: \R \to \R, x \mapsto x^2$ ist stetig, aber nicht nicht gleichmässig stetig.
\end{definition}

\begin{satz}{5.15}[Heine]
    Sei $f: [a,b] \to \R$ stetig in dem kompakten Intervall $[a,b]$. Dann ist $f$ in $[a,b]$ gleichmässig stetig.
\end{satz}

\begin{satz}{5.16}[Integrierbar/Stetig]
    Sei $f: [a,b] \to \R$ stetig. Dann ist $f$ integrierbar.
\end{satz}

\begin{satz}{5.17}
    Sei $f: [a,b] \to \R$ monoton. Dann ist $f$ integrierbar.
\end{satz}

\begin{satz}{5.19}
    Sei $I \subsetneq \R$ ein kompaktes Intervall mit Endpunkten $a, b$ sowie $f_1, f_2: I \to \R$ beschränkt integrierbar
    und $\lambda_1,\lambda_2 \in \R$. Dann gilt
    \[ \int_a^b (\lambda_1 f_1(x) | \lambda_2 f_2(x)) dx = \lambda_1 \int_a^b f_1(x)dx + \lambda_2 \int_a^b f_2(x) dx \]
\end{satz}

\begin{satz}{5.20}
    $f, g: [a,b] \to \R$ beschränkt integrierbar, und $f(x) \le g(x)\ \forall x \in [a, b]$.
    Dann folgt
    \[ \int_a^b f(x)dx \le \int_a^b g(x)dx \]
\end{satz}

\begin{satz}{5.22}[Cauchy 1821, Schwarz 1885, Bunjakovski 1859]
    Seien $f,g: [a,b] \to \R$ beschränkt integrierbar. Dann gilt
    \[ \left| \int_a^b f(x)g(x) dx \right| \le \sqrt{\int_a^b f^2(x)dx} \sqrt{\int_a^b g^2(x)dx} \]
\end{satz}

\begin{satz}{5.23}[Mittelwertsatz, Cauchy 1821]
    Sei $f: [a,b] \to \R$ stetig. Dann gibt es $\xi \in [a,b]$ mit
    \[ \int_a^b f(x) dx = f(\xi)(b-a) \]
\end{satz}

\subsection{Differentialrechnung}

\begin{satz}{5.26}
    Seien $a < b$ und $f: [a, b] \to \R$ stetig. Die Funktion
    \[ F(x) = \int_a^x f(t) dt,\quad a \le x \le b \]
    ist int $[a, b]$ stetig differenzierbar und
    \[ F'(x) = f(x, \quad \forall x \in [a, b]) \]
\end{satz}

\begin{definition}{5.27}
    Sei $a < b$ und $f: [a,b] \to \R$ stetig. Eine Funktion $F: [a,b ] \to \R$ heisst \textbf{Stammfunktion} von $f$, falls $F$
    (stetig) differenzierbar in $[a, b]$ ist und $F' = f$ in $[a, b]$ gilt.
\end{definition}

\begin{satz}{5.28}[Fundamentalsatz der Differenzialrechnung]
    Sei $f: [a,b] \to \R$ stetig. Dann gibt es eine Stammfunktion $F$ von $f$, die bis auf eine additive Konstante eindeutig bestimmt ist und es gilt:
    \[ \int_a^b f(x)dx = F(b) - F(a) \]
\end{satz}

\begin{satz}{5.30}[Partielle Integration]
    Seie $a<b$ relle Zahlen und $f,g: [a,b] \to \R$ stetig differenzierbar. Dann gilt
    \[ \int_a^b f(x)g'(x)dx = f(b)g(b) - f(a)g(a) - \int_a^b f'(x)g(x)dx \]
\end{satz}

\begin{satz}{5.31}[Substitution]
    sei $a < b$, $\phi: [a, b] \to \R$ stetig differenzierbar, $I \subseteq  \R$ ein Intervall mit $\phi([a, b]) \subseteq I$ und $f: I \to \R$ eine stetige Funktion.
    Dann gilt
    \[ \int_{\phi(a)}^{\phi(b)} f(x)dx = \int_a^b f(\phi(t)) \phi'(t) dt \]
\end{satz}

\subsection{Integration konvergenter Reihen}
\begin{satz}{5.34}
    Sei $f_n: [a, b] \to \R$ eine Folge von beschränkten, integrierbaren Funktionen die gleichmässig gegen eine Funktion $f: [a,b] \to \R$ konvergiert.
    Dann ist $f$ beschränkt integrierbar und
    \[ \lim_{n \to \infty} \int_a^b f_n(x)dx = \int_a^b f(x)dx \]
\end{satz}

\subsection{Euler-McLaurinSummationsformel}
\todo[inline]{maybe useful? maybe not?}

\subsection{Uneigentliche Integrale}
\begin{definition}
    Sei $f: [a, \infty[ \to \R$ beschränkt und integrierbar auf $[a,b]$ für alle $b > a$. Falls
    \[ \lim_{b \to \infty} \int_a^b f(x) dx \]
    existiert, bezeichnen wir den Grenzwert mit
    \[ \int_a^\infty f(x)dx \]
    und sagen, dass $f$ auf $[a, \infty[$ integrierbar ist.
\end{definition}

\begin{satz}{5.53}[McLaurin 1742]
    Sei $f: [1, \infty[ \to [0, \infty[$ monoton fallend. Die Reihe
    \[ \sum_{n=1}^\infty f(n) \]
    konvergiert genau dann, wenn
    \[ \int_a^\infty f(x)dx \]
    konvergiert.
\end{satz}

\begin{definition}{5.56}[NATO caliber]
    In dieser Situation ist $f: ]a, b] \to \R$ integrierbar, falls
    \[ \lim_{\epsilon \to 0^+} \int_{a + \epsilon}^b f(x)dx \]
    existiert. In diesem Fall wird den Grenzwert mit
    $\int_a^b f(x)dx$ bezeichnet
\end{definition}

\subsection{Die Gamma Funktion}

\begin{definition}{5.59}
    Für $s > 0$ definieren wir
    \[ \Gamma(s) := \int_0^\infty e^{-x} x^{s - 1} dx \]

    Konvergiert für alle $s > 0$
\end{definition}


\begin{satz}{5.60}[Bohr-Mollerup]
    \begin{enumerate}
        \item Die Gamma Funktion erfüllt die Relationen
              \begin{enumerate}
                  \item $\Gamma(1) = 1$
                  \item $\Gamma(s + 1) = s\Gamma(s)\ \forall s > 0$
                  \item $\Gamma$ ist logarithmisch konvex, das heisst $\Gamma(\lambda x + (1 - \lambda)y) \le \Gamma(x)^\lambda \Gamma(y)^{1 - \lambda}$
                        für alle $x,y > 0$ und $0 \le \lambda \le 1$
              \end{enumerate}
        \item Die Gamma Funktion ist die einzige Funktion $]0, \infty[ \to ]0, \infty[$, die die drei Relationen erfüllt.
              Darüber hinaus gilt:
              \[ \Gamma(x) = \lim_{n \to +\infty} \frac{n! n^x}{x(x+1)...(x+n)}\ \forall x > 0 \]
    \end{enumerate}
\end{satz}

\begin{lemma}{5.61}
    Sei $p > 1$ und $q > 1$ mit $\frac{1}{p} + \frac{1}{q} = 1$.
    Dann gilt $\forall a,b \ge 0$
    \[ a \cdot b \le \frac{a^p}{p} + \frac{b^q}{q} \]
\end{lemma}

\begin{satz}{5.62}[Hölder Ungleichung]
    Seien $p > 1$ und $q > 1$ mit $\frac{1}{p} + \frac{1}{q} = 1$.
    Für alle $f, g: [a,b] \to \R$ stetig gilt
    \[ \int_a^b |f(x)g(x)|dx \le ||f||_p||g||_q \]

    Wobei
    \[ ||f||_p := \left( \int_a^b |f(x)|^p dx \right)^{\frac{1}{p}} \]
\end{satz}

\subsection{Das unbestimmte Integral}

Sei $f: I \to \R$ auf einem Intervall $I \subseteq \R$ definiert. Falls $f$ stetig ist, gibt es eine Stammfunktion $F$ für $f$. Wir schreiben
\[ \int f(x)dx = F(x) + C \]

\subsubsection{Stammfunktionen von rationalen Funktionen}

\section{Binomialsatz}
\begin{satz}{A.1}[Binomialsatz]
    Für alle $x,y \in \C,\ n \ge 1$ gilt:
    \[ (x + y)^n = \sum_{k=0}^n \binom{n}{k} x^k y^{n - k} \]
\end{satz}

\section{Derivative Cheat Sheet}

Properties
\begin{alignat*}{2}
     & (cf)' = c f'(x)     & (f \pm g)' = f'(x) \pm g'(x)                       \\
     & (fg)' = f'g + fg'   & (\frac{f}{g})' = \frac{f'g - fg'}{g^2}             \\
     & \frac{d}{dx}(c) = 0 & \frac{d}{dx}\left( g(f(x)) \right) = g'(f(x))f'(x)
\end{alignat*}

\begin{tabular}{|c | c |}
    \hline
    \rowcolor{lightgray} $f(x)$ & $f'(x)$                       \\
    \hline
    $x$                         & $1$                           \\
    \hline
    $x^n$                       & $nx^{n-1}$                    \\
    \hline
    $\sin x$                    & $\cos x$                      \\
    \hline
    $ \cos x $                  & $ -\sin x $                   \\
    \hline
    $ \tan x $                  & $ \sec^2 x $                  \\
    \hline
    $ \sec x $                  & $ \sec x \tan x $             \\
    \hline
    $ \csc x $                  & $ -\csc x \cot x$             \\
    \hline
    $ \cot x $                  & $ -\csc^2 x $                 \\
    \hline
    $ \arcsin x$                & $ \frac{1}{\sqrt{1 - x^2}} $  \\
    \hline
    $ \arccos x$                & $ -\frac{1}{\sqrt{1 - x^2}} $ \\
    \hline
    $ \arctan x$                & $ \frac{1}{1 + x^2} $         \\
    \hline
    $ a^x $                     & $ a^x \ln(a) $                \\
    \hline
    $ e^x $                     & $ e^x $                       \\
    \hline
    $ \ln(x) $                  & $ \frac{1}{x},\ x > 0 $       \\
    \hline
    $ \ln(|x|) $                & $ \frac{1}{x},\ x \ne 0 $     \\
    \hline
    $ \log_a(x) $               & $ \frac{1}{x \ln a},\ x > 0 $ \\
\end{tabular}

\section{Integral Cheat Sheet}

\[ \int f(x)dx = F(x) + C \]

\begin{tabular}{| c | c |}
    \hline
    \rowcolor{lightgray} $f(x)$ & $F(X)$ (without $+ C)$               \\
    \hline
    $x^s$                       & $\frac{x^{s + 1}}{s + 1},\ s \ne -1$ \\
    \hline
    $x^s$                       & $\ln x,\ x \ge 0$                    \\
    \hline
    $e^x$                       & $e^x$                                \\
    \hline
    $\sin x$                    & $-\cos x$                            \\
    \hline
    $\cos x$                    & $\sin x$                             \\
    \hline
    $\sinh x$                   & $\cosh x$                            \\
    \hline
    $\cosh x$                   & $\sinh x$                            \\
    \hline
    $\frac{1}{\sqrt{1 - x^2}}$  & $\arcsin x$                          \\
    \hline
    $\frac{1}{1 + x^2}$         & $\arctan x$                          \\
    \hline
    $\frac{1}{\sqrt{1 + x^2}}$  & $\asinh x$                           \\
    \hline
    $\frac{1}{\sqrt{x^2 - 1}}$  & $\acosh x$                           \\
    \hline
\end{tabular}

\todo[inline]{More?}

\end{document}