\documentclass[8pt,a4paper,twocolumn]{extarticle}
\usepackage{mathtools}
\usepackage{amsfonts}
\usepackage{amsthm}
\usepackage[margin=0.5cm]{geometry}
\usepackage{hyperref}


\setlength{\columnsep}{1cm}
\setlength{\columnseprule}{0.2pt}

\newcommand{\N}{\mathbb{N}}
\newcommand{\R}{\mathbb{R}}


\newtheorem{innercustomgeneric}{\customgenericname}
\providecommand{\customgenericname}{}
\newcommand{\newcustomtheorem}[2]{%
  \newenvironment{#1}[1]
  {%
   \renewcommand\customgenericname{#2}%
   \renewcommand\theinnercustomgeneric{##1}%
   \innercustomgeneric
  }
  {\endinnercustomgeneric}
}

\newcustomtheorem{definition}{Definition}

\newcustomtheorem{lemma}{Lemma}

\newcustomtheorem{satz}{Satz}

\newcustomtheorem{korollar}{Korollar}

\newcommand{\seq}[1]{\left( #1_n \right)_{n \ge 1}}

\title{Analysis 1 Recap}
\author{Axel Montini \\ \href{mailto:amontini@student.ethz.ch}{amontini@student.ethz.ch}}

\begin{document}
\maketitle

\section{Start}

\begin{korollar}{1.16}
    \label{korollar:inf-sup}
    $A \subseteq B \subseteq \R$, dann
    \begin{itemize}
        \item $B$ nach oben beschraenkt $\Rightarrow \sup A \le \sup B$
        \item $B$ nach unten beschraenkt $\Rightarrow \inf B \le \inf A$
    \end{itemize}
\end{korollar}

\section{Folgen und Reihen}
\subsection{Folgen}
\begin{definition}{2.4}
    $\left\{ n \in \N : a_n \notin \left] l - \epsilon, l + \epsilon \right[ \right\}$
    $\Rightarrow \seq{a}\ \mbox{heisst \textbf{konvergent}}$
\end{definition}

\begin{definition}{2.4+}
    Grenzwert oder Limes der Folge: $\underset{n \to \infty}{\lim} a_n$
\end{definition}

\begin{lemma}
    Equivalent Aussagen:
    \begin{enumerate}
        \item $\seq{a}$ konvergiert gegen $l = \lim_{n \to \infty} a_n$
        \item $\forall \epsilon > 0\ \exists N \ge 1\ \left( \lvert a_n - l \rvert < \epsilon \quad \forall n \ge N \right)$
    \end{enumerate}
\end{lemma}

\begin{satz}{2.8}
    Konvergente Folgen $\seq{a},\ \seq{b},\ a = \lim_{n \to \infty} a_n,\ b = \lim_{n \to \infty} b_n$
    \begin{enumerate}
        \item $(a_n + b_n)_{n \ge 1}$ auch konvergent und $a + b = \lim_{n \to \infty} (a_n + b_n)$
        \item $(a_n \cdot b_n)_{n \ge 1}$ konvergiert und $a \cdot b = \lim \dots$
        \item $\forall n \ge 1\ (b_n \ne 0)$ and $b \ne 0$. Dann $\left(\frac{a_n}{b_n}\right)_{n \ge 1}$ konvergent und $\frac{a}{b} = \lim \dots$
        \item $\exists K \ge 1\ \left(\forall n \ge K\ (a_n \le b_n)\right) \Rightarrow a \le b$
    \end{enumerate}
\end{satz}

\begin{definition}{2.10}
    \[\forall n \ge 1\ (a_n \le a_{n + 1}) \Rightarrow \seq{a} \mbox{ist \textbf{monoton wachsend}}\]
    \[\forall n \ge 1\ (a_n \ge a_{n + 1}) \Rightarrow \seq{a} \mbox{ist \textbf{monoton fallend}}\]
\end{definition}

\begin{satz}{2.11}
    \par \textbf{Weierstrass}
    \begin{itemize}
        \item $\seq{a}$ monoton wachsend und nach oben beschraenkt. Dann konvergiert mit $\lim_{n \to \infty} a_n = \sup \left\{ a_n : n \ge 1 \right\}$
        \item $\seq{a}$ monoton fallend und nach unten beschraenkt. Dann konvergiert mit $\lim_{n \to \infty} a_n = \inf \left\{ a_n : n \ge 1 \right\}$
    \end{itemize}
\end{satz}

\begin{lemma}{}
    Bernoulli Ungleichung: $(1 + x)^n \ge 1 + nx\ \forall n \in \N,\ x > -1$
\end{lemma}

\begin{definition}
    \par Limes superior und limes inferior.
    Folge $\seq{a}$ + Weierstrass, definiert man zwei monotone Folgen $\seq{b}$ und $\seq{c}$.
    Fuer jedes $n \ge 1$: $b_n = \inf \left\{ a_k : k \ge n \right\}$ und $c_n = \sup \left\{ a_k : k \ge n \right\}$.
    Dann folgt aus Korollar \ref{korollar:inf-sup}:
    $b_n \le b_{n + 1}\ \forall n \ge 1$ und $c_n \ge c_{n + 1}\ \forall n \ge 1$ (beide konvergieren).

    \begin{itemize}
        \item Limes inferior: $\underset{n \to \infty}{\lim \inf} a_n := \lim_{n \to \infty} b_n$
        \item Limes superior: $\underset{n \to \infty}{\lim \sup} a_n := \lim_{n \to \infty} c_n$
    \end{itemize}\

    Aus $b_n \le c_n$ folgt $\liminf_{n \to \infty} a_n \le \limsup_{n \to \infty} a_n$
\end{definition}

\begin{lemma}{2.19}
    \[
        \seq{a}\ \mbox{konv.} \Leftrightarrow \seq{a}\ \mbox{besch. und}\ \liminf_{n \to \infty} a_n = \limsup_{n \to \infty} a_n
    \]
\end{lemma}

\begin{satz}{2.20}
    \label{satz:cauchy-criterium-folge}
    \par Cauchy Criterium
    \[ \seq{a}\ \mbox{konvergent} \Leftrightarrow \forall \epsilon > 0\
    \exists N \ge 1\ \left( \forall n,m \ge N \left( \lvert a_n - a_m \rvert < \epsilon \right) \right) \]
\end{satz}

\begin{definition}{2.21}
    \par Abgeschlossenes Intervall (Teilmenge $I \subseteq \R$)
    Form $[a,b]$ oder $[a, +\infty[$ oder $]-\infty, a]$ oder $]-\infty, +\infty[$ mit $a \le b,\ a,b \in \R$.

    Laenge der intervalls: $\mathcal{L}(I) = b - a$ in Fall 1, $\mathcal{L}(I) = +\infty$ anderfalls.
\end{definition}

\begin{satz}{2.25}
    \label{satz:cauchy-cantor}
    \par Cauchy-Cantor

    $I_1 \supseteq I_2 \supseteq \dots \supseteq I_n \supseteq I_{n + 1} \dots$ Folge abgeschlossener Intervalle mit $\mathcal{L}(I_1) < +\infty$.
    Dann gilt $\bigcap_{n \ge 1} I_n \ne \emptyset$.

    Falls $\underset{n \to \infty}{\lim} \mathcal{L}(I_n) = 0$ enthaelt $\bigcap_{n \ge 1} I_n$ genau einer punkt.
\end{satz}

\begin{definition}{2.27}
    \par Teilfolge
    Eine Teilfolge von $\seq{a}$ ist $\seq{b}$ wobei $b_n = a_{l(n)}$ und $l: \N^* \to \N^*$ satisfies $l(n) \le l(n + 1)\ \forall n \ge 1$
\end{definition}

\begin{satz}{2.29}
    \label{satz:bolzano-weierstrass}
    \par Bolzano-Weierstrass

    Jede beschraenkte Folge besitzt eine konvergente Teilfolge.
\end{satz}

\subsection{Reihen}
Sei $\seq{a}$ eine Folge in $\R$ oder $\mathbb{C}$. Der Begriff der Konvergenz der Reihe $\sum_{k = 1}^{\infty} a_k$ stuetzt sich auf die
Folge $\seq{S}$ der Partialsummen $\sum_{k = 1}^n a_k$

\begin{definition}{2.37}
    Die reihe $\sum_{k = 1}^{\infty} a_k$ ist \textbf{konvergent}, falls die Folge $\seq{S}$ der Partialsummen konvergiert.
    Definieren wir $\sum_{k = 1}^{\infty} a_k := \lim_{n \to \infty} S_n$
\end{definition}

\begin{satz}{2.40}
    Seien $\sum_{k=1}^{\infty} a_k$ und $\sum_{j=1}^{\infty} b_j$ konvergent, sowie $\alpha \in \mathbb{C}$.
    Dann:
    \begin{enumerate}
        \item $\sum_{k=1}^{\infty} (a_k + b_k) = \left( \sum_{k=1}^{\infty} a_k \right) + \left( \sum_{j=1}^{\infty} b_j \right)$ ist konvergent.
        \item $\sum_{k=1}^{\infty} \alpha \cdot a_k = \alpha \cdot \sum_{k=1}^{\infty} a_k$ ist konvergent.
    \end{enumerate}
\end{satz}

\begin{satz}{2.41}
    \label{satz:cauchy-criterium-reihe}
    \par Cauchy Criterium, konvergente Reihe
    \[
        \sum_{k=1}^\infty a_k\ \mbox{konvergent}
        \Leftrightarrow
        \forall \epsilon > 0\ \exists N \ge 1\ \left( \left| \sum_{k=n}^m a_k \right| < \epsilon,\ \forall m \ge n \ge N \right)
    \]
\end{satz}

\begin{satz}{2.42}
    \[
        (a_k \ge 0\ \forall k \in \N^*)
        \quad 
        \sum_{k=1}^\infty a_k\ \mbox{konvergent}
        \Leftrightarrow
        \seq{S}\ \mbox{nachoben beschraenkt}
    \]
\end{satz}

\begin{korollar}{2.43} Vergleichssatz
    \begin{align*}
        b_k \ge a_k \ge 0\ \forall k \ge 1
        \Rightarrow \begin{cases}
            \sum_{k=1}^\infty b_k\ \mbox{konvergent} &\Rightarrow \sum_{k=1}^\infty a_k\ \mbox{konvergent}\\
            \sum_{k=1}^\infty a_k\ \mbox{divergent} &\Rightarrow \sum_{k=1}^\infty b_k\ \mbox{divergent}
        \end{cases}
    \end{align*}
\end{korollar}

\begin{definition}{2.45}
    \[ \sum_{k=1}^\infty \left| a_k \right|\ \mbox{konvergiert} \Rightarrow  \sum_{k=1}^\infty a_k\ \mbox{\textbf{konvergiert absolut}} \]
\end{definition}

\begin{satz}{2.46}
    Eine absolut kovergente Reihe ist auch konvergent und es gilt $\left| \sum_{k=1}^\infty a_k \right| \le \sum_{k=1}^\infty \left| a_k \right|$
\end{satz}

\begin{satz}{2.48} \par Leibniz 1682
    \begin{align*}
        &\seq{a}\ \mbox{monoton fallend}, a_n \ge 0, \lim_{n \to \infty} a_n = 0\\
        &\Rightarrow S := \sum_{k=1}^\infty (-1)^{k+1} a_k \ \mbox{konvergiert und es gilt}\ a_1 - a_2 \le S \le a_1
    \end{align*}
\end{satz}

\begin{definition}{2.50} \par Umordnung
    \[
    \exists \phi: \N^* \to \N^*\ \left( \phi\ \mbox{bijek.,}\ a'_n = a_{\phi(n)} \right)    
    \Rightarrow
    \sum_{n=1}^\infty a'_n\ \mbox{\textbf{Umordnung} von}\ \sum_{n=1}^\infty a_n
    \]
\end{definition}

\begin{satz}{2.52} (Dirichlet 1837)
    Falls $\sum_{n=1}^\infty a_n$ absolut konvergiert, denn konvergiert jede Umordnung der Reihe und hat denselben Grenzwert.
\end{satz}

\begin{satz}{2.53} (Quotientenkriterium, Cauchy 1821)
    \begin{align*}
        &\seq{a}, a_n \ne 0\ \forall n \ge 1\\
            &\limsup_{n \to \infty} \frac{|a_{n+1}|}{|a_n|} < 1 \Rightarrow \sum_{n=1}^\infty a_n\ \mbox{konvergiert absolut}\\
            &\liminf_{n \to \infty} \frac{|a_{n+1}|}{|a_n|} > 1 \Rightarrow \sum_{n=1}^\infty a_n\ \mbox{divergiert}
    \end{align*}
\end{satz}

\begin{satz}{2.56} (Wurzelkriterium, Cauchy 1821)
\begin{align*}
    \limsup_{n \to \infty} \sqrt[n]{|a_n|} < 1 &\Rightarrow \sum_{n=1}^\infty a_n\ \mbox{konvergiert absolut}\\
    \limsup_{n \to \infty} \sqrt[n]{|a_n|} > 1 &\Rightarrow \sum_{n=1}^\infty a_n, \sum_{n=1}^\infty |a_n|\ \mbox{divergieren}
\end{align*}
\end{satz}

\begin{korollar}{2.57}
    $\sum_{k=0}^\infty c_k z^k$ konvergiert absolut fuer alle $|z| < \rho$ und divergiert fuer alle $|z| > \rho$,
    mit $\rho := \left( \limsup_{k \to \infty} \sqrt[k]{|c_k|} \right)$ (oder $\rho = +\infty$ wenn denominatore 0).
\end{korollar}

\begin{korollar}{}
    Die zeta funktion konvergiert, $\zeta(s) = \sum_{n=1}^\infty \frac{1}{n^s}$
\end{korollar}

\begin{definition}{2.58}
    $\sum_{k=0}^\infty b_k$ ist eine \textbf{lineare Anordnung} der Doppelreihe $\sum_{i,j \ge 0} a_{ij}$ falls es
    eine Bijektion $\sigma : \N \to \N \times \N$ gibt mit $b_k = a_{\sigma(k)}$
\end{definition}

\begin{satz}{2.59} (Cauchy 1821)
    Wir nehmen an dass $\exists B \ge 0$ so dass
    \[ \sum_{i=0}^m \sum_{j=0}^m |a_{ij}| \le B\ \forall m \ge 0 \]
    Dann konvergieren absolut:
    \[
        S_i := \sum_{j=0}^\infty a_{ij}\ \forall i \ge 0,
        U_j := \sum_{i=0}^\infty a_{ij}\ \forall j \ge 0,
        \sum_{i=0}^\infty S_i,
        \sum_{j=0}^\infty U_j,
    \]
    und es gilt $\sum_{j=0}^\infty U_j = \sum_{i=0}^\infty S_i$.
    Jede lineare Anordnung der Doppelreihe konvergiert absolut, mit selbem Grenzwert.
\end{satz}
\end{document}