\section{Riemann Integral}

\begin{definition}{5.1}
    Eine \textbf{Partition} von $I$ ist eine endliche Teilmenge $P \subsetneq [a, b]$ wobei $\{a,b\} \subseteq P$.
\end{definition}

\begin{definition}{}[Untersumme, Obersumme]
    \begin{alignat*}{2}
        \mbox{Partition}\  & P = \{x_0, x_1, ..., x_n\}\           & \mbox{mit}\ x_0 = a < x_1 < ... < x_n = b \\
                           & \delta_i = x_i - x_{i - 1},           & i \ge 1                                   \\
                           & s(f, P) := \sum_{i=1}^n f_i \delta_i, & f_i = \inf_{x_{i - 1} \le x \le x_i} f(x) \\
                           & S(f, P) := \sum_{i=1}^n F_i \delta_i, & F_i = \sup_{x_{i - 1} \le x \le x_i} f(x) \\
    \end{alignat*}
\end{definition}

\begin{lemma}{5.2}
    \begin{enumerate}
        \item Sei $P'$ eine Vereinfachung von $P$, dann gilt $s(f, P) \le s(f, P') \le S(f, P') \le S(f, P)$
        \item Für beliebige Partitionen $P_1, P_2$ gilt: $s(f, P_1) \le S(f, P_2)$
    \end{enumerate}
\end{lemma}

\begin{definition}{}
    Sei $\mathcal{P}(I)$ der Menge der Partitionen von $I$.
    \begin{align*}
        s(f) & := \sup_{P \in \mathcal{P}(I)} s(f, P) \\
        S(f) & := \inf_{P \in \mathcal{P}(I)} S(f, P)
    \end{align*}
\end{definition}

\begin{satz}{5.3}
    Eine beschränkte Funktion $f: [a,b] \to \R$ ist \textbf{Riemann integrierbar} (oder kurz: integrierbar) falls
    \[ s(f) = S(f) \]
    In diesem fall bezeichnen wir den gemeinsamen Wert von $s(f)$ und $S(f)$ mit
    \[ \int_a^b f(x) dx \]
\end{satz}

\begin{satz}{5.4}
    Eine beschränkte Funkiton ist genau dann integrierbar falls
    \[ \forall \epsilon > 0\ \exists P \in \mathcal{P}(I)\ \mbox{mit}\ S(f, P) - s(f, P) \le \epsilon \]
\end{satz}

\begin{satz}{5.8}[Du Bois-Reymond 1875, Darboux 1875]
    Eine beschränkte Funktion $f: [a, b] \to \R$ ist genau dann integrierbar, falls $\forall \epsilon > 0\ \exists \delta > 0$
    so dass
    \[ \forall P \in \mathcal{P}_\delta(I),\ S(f, P) - s(f, P) < \epsilon \]
\end{satz}

\begin{satz}{5.10}
    Seien $f, g: [a, b] \to \R$ beschränkt, integrierbar und $\lambda \in R$. Dann sind $f + g$, $\lambda \cdot f$, $f \cdot g$, $|f|$,
    $\max(f, g)$, $\min(f, g)$ und $\frac{f}{g}$ (falls $|g(x)| \ge \beta > 0\ \forall x \in [a, b]$) integrierbar
\end{satz}

\begin{definition}
    Eine funktion $f: D \to \R$ ist \textbf{gleichmässig stetig}, falls
    \[ \forall \epsilon > 0\ \exists \delta > 0\ \forall x,y \in D\ \left( |x - y| < \delta \implies |f(x) - f(y)| < \epsilon \right) \]

    Example: $f: \R \to \R, x \mapsto x^2$ ist stetig, aber nicht nicht gleichmässig stetig.
\end{definition}

\begin{satz}{5.15}[Heine]
    Sei $f: [a,b] \to \R$ stetig in dem kompakten Intervall $[a,b]$. Dann ist $f$ in $[a,b]$ gleichmässig stetig.
\end{satz}

\begin{satz}{5.16}[Integrierbar/Stetig]
    Sei $f: [a,b] \to \R$ stetig. Dann ist $f$ integrierbar.
\end{satz}

\begin{satz}{5.17}
    Sei $f: [a,b] \to \R$ monoton. Dann ist $f$ integrierbar.
\end{satz}

\begin{satz}{5.19}
    Sei $I \subsetneq \R$ ein kompaktes Intervall mit Endpunkten $a, b$ sowie $f_1, f_2: I \to \R$ beschränkt integrierbar
    und $\lambda_1,\lambda_2 \in \R$. Dann gilt
    \[ \int_a^b (\lambda_1 f_1(x) | \lambda_2 f_2(x)) dx = \lambda_1 \int_a^b f_1(x)dx + \lambda_2 \int_a^b f_2(x) dx \]
\end{satz}

\begin{satz}{5.20}
    $f, g: [a,b] \to \R$ beschränkt integrierbar, und $f(x) \le g(x)\ \forall x \in [a, b]$.
    Dann folgt
    \[ \int_a^b f(x)dx \le \int_a^b g(x)dx \]
\end{satz}

\begin{satz}{5.22}[Cauchy 1821, Schwarz 1885, Bunjakovski 1859]
    Seien $f,g: [a,b] \to \R$ beschränkt integrierbar. Dann gilt
    \[ \left| \int_a^b f(x)g(x) dx \right| \le \sqrt{\int_a^b f^2(x)dx} \sqrt{\int_a^b g^2(x)dx} \]
\end{satz}

\begin{satz}{5.23}[Mittelwertsatz, Cauchy 1821]
    Sei $f: [a,b] \to \R$ stetig. Dann gibt es $\xi \in [a,b]$ mit
    \[ \int_a^b f(x) dx = f(\xi)(b-a) \]
\end{satz}

\subsection{Differentialrechnung}

\begin{satz}{5.26}
    Seien $a < b$ und $f: [a, b] \to \R$ stetig. Die Funktion
    \[ F(x) = \int_a^x f(t) dt,\quad a \le x \le b \]
    ist int $[a, b]$ stetig differenzierbar und
    \[ F'(x) = f(x, \quad \forall x \in [a, b]) \]
\end{satz}

\begin{definition}{5.27}
    Sei $a < b$ und $f: [a,b] \to \R$ stetig. Eine Funktion $F: [a,b ] \to \R$ heisst \textbf{Stammfunktion} von $f$, falls $F$
    (stetig) differenzierbar in $[a, b]$ ist und $F' = f$ in $[a, b]$ gilt.
\end{definition}

\begin{satz}{5.28}[Fundamentalsatz der Differenzialrechnung]
    Sei $f: [a,b] \to \R$ stetig. Dann gibt es eine Stammfunktion $F$ von $f$, die bis auf eine additive Konstante eindeutig bestimmt ist und es gilt:
    \[ \int_a^b f(x)dx = F(b) - F(a) \]
\end{satz}

\begin{satz}{5.30}[Partielle Integration]
    Seie $a<b$ relle Zahlen und $f,g: [a,b] \to \R$ stetig differenzierbar. Dann gilt
    \[ \int_a^b f(x)g'(x)dx = f(b)g(b) - f(a)g(a) - \int_a^b f'(x)g(x)dx \]
\end{satz}

\begin{satz}{5.31}[Substitution]
    sei $a < b$, $\phi: [a, b] \to \R$ stetig differenzierbar, $I \subseteq  \R$ ein Intervall mit $\phi([a, b]) \subseteq I$ und $f: I \to \R$ eine stetige Funktion.
    Dann gilt
    \[ \int_{\phi(a)}^{\phi(b)} f(x)dx = \int_a^b f(\phi(t)) \phi'(t) dt \]
\end{satz}

\subsection{Integration konvergenter Reihen}
\begin{satz}{5.34}
    Sei $f_n: [a, b] \to \R$ eine Folge von beschränkten, integrierbaren Funktionen die gleichmässig gegen eine Funktion $f: [a,b] \to \R$ konvergiert.
    Dann ist $f$ beschränkt integrierbar und
    \[ \lim_{n \to \infty} \int_a^b f_n(x)dx = \int_a^b f(x)dx \]
\end{satz}

\subsection{Euler-McLaurinSummationsformel}
\todo[inline]{maybe useful? maybe not?}

\subsection{Stirling'sche Formel}
\begin{satz}{5.47}
    \[ n! = \frac{\sqrt{2 \pi n} n^n}{e^n} \cdot \exp \left( \frac{1}{12n} + R_3(n) \right) \]
    wobei
    \[ |R_3(n)| \le \frac{\sqrt{3}}{216} \cdot \frac{1}{n^2}\quad \forall n \ge 1 \]
\end{satz}

\begin{satz}{}[Approx $n!$]
    \[ n! \approx \frac{\sqrt{2 \pi n}n^n}{e^n} \]
    and
    \[ \lim_{n \to \infty} n! \ \frac{\sqrt{2 \pi n}n^n}{e^n} = 1 \]
\end{satz}

\subsection{Uneigentliche Integrale}
\begin{definition}
    Sei $f: [a, \infty[ \to \R$ beschränkt und integrierbar auf $[a,b]$ für alle $b > a$. Falls
    \[ \lim_{b \to \infty} \int_a^b f(x) dx \]
    existiert, bezeichnen wir den Grenzwert mit
    \[ \int_a^\infty f(x)dx \]
    und sagen, dass $f$ auf $[a, \infty[$ integrierbar ist.
\end{definition}

\begin{satz}{5.53}[McLaurin 1742]
    Sei $f: [1, \infty[ \to [0, \infty[$ monoton fallend. Die Reihe
    \[ \sum_{n=1}^\infty f(n) \]
    konvergiert genau dann, wenn
    \[ \int_a^\infty f(x)dx \]
    konvergiert.
\end{satz}

\begin{definition}{5.56}[NATO caliber]
    In dieser Situation ist $f: ]a, b] \to \R$ integrierbar, falls
    \[ \lim_{\epsilon \to 0^+} \int_{a + \epsilon}^b f(x)dx \]
    existiert. In diesem Fall wird den Grenzwert mit
    $\int_a^b f(x)dx$ bezeichnet
\end{definition}

\subsection{Die Gamma Funktion}

\begin{definition}{5.59}
    Für $s > 0$ definieren wir
    \[ \Gamma(s) := \int_0^\infty e^{-x} x^{s - 1} dx \]

    Konvergiert für alle $s > 0$
\end{definition}


\begin{satz}{5.60}[Bohr-Mollerup]
    \begin{enumerate}
        \item Die Gamma Funktion erfüllt die Relationen
              \begin{enumerate}
                  \item $\Gamma(1) = 1$
                  \item $\Gamma(s + 1) = s\Gamma(s)\ \forall s > 0$
                  \item $\Gamma$ ist logarithmisch konvex, das heisst $\Gamma(\lambda x + (1 - \lambda)y) \le \Gamma(x)^\lambda \Gamma(y)^{1 - \lambda}$
                        für alle $x,y > 0$ und $0 \le \lambda \le 1$
              \end{enumerate}
        \item Die Gamma Funktion ist die einzige Funktion $]0, \infty[ \to ]0, \infty[$, die die drei Relationen erfüllt.
              Darüber hinaus gilt:
              \[ \Gamma(x) = \lim_{n \to +\infty} \frac{n! n^x}{x(x+1)...(x+n)}\ \forall x > 0 \]
    \end{enumerate}
\end{satz}

\begin{lemma}{5.61}
    Sei $p > 1$ und $q > 1$ mit $\frac{1}{p} + \frac{1}{q} = 1$.
    Dann gilt $\forall a,b \ge 0$
    \[ a \cdot b \le \frac{a^p}{p} + \frac{b^q}{q} \]
\end{lemma}

\begin{satz}{5.62}[Hölder Ungleichung]
    Seien $p > 1$ und $q > 1$ mit $\frac{1}{p} + \frac{1}{q} = 1$.
    Für alle $f, g: [a,b] \to \R$ stetig gilt
    \[ \int_a^b |f(x)g(x)|dx \le ||f||_p||g||_q \]

    Wobei
    \[ ||f||_p := \left( \int_a^b |f(x)|^p dx \right)^{\frac{1}{p}} \]
\end{satz}

\subsection{Das unbestimmte Integral}

Sei $f: I \to \R$ auf einem Intervall $I \subseteq \R$ definiert. Falls $f$ stetig ist, gibt es eine Stammfunktion $F$ für $f$. Wir schreiben
\[ \int f(x)dx = F(x) + C \]

\subsubsection{Stammfunktionen von rationalen Funktionen}

